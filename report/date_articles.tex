Having some metrics to compare the articles between years, we thought it was a good idea to try to date a subset of articles from a year with these metrics. Thus, we wrote a \emph{Scala} application that selects a subset of articles in a year.\\
Nevertheless, we had to adapt a bit our codes because it was unnecessary to compare years with years but only articles with years.\\
To simulate the fact that the selected subset of articles is new and cannot be taken into account in the 'already known' corpus, we had to remove the articles from the year they belong to.\\
Finally, some titles of articles in the newspaper are considered as a complete article. So we selected a subset of articles and not only a single one to avoid having the impossible job of dating only 4 or 5 words.\\
The dating is done as follows: the distance is computed between the selection of articles and each year of the corpus. The prediction is then given by the year for which the distance is the smallest.\\

You can find below the results for the different metrics with the years 1880, 1920 and 1995 and with taking a subset of 15 articles in each year.

\subsection{Metrics}

\subsubsection{Basic Distance}
The dating is done by computing the distance between a small set of articles and each year. The prediction is the year for which the distance is the smallest.

\begin{figure}[H]
    \begin{minipage}[b]{0.3\linewidth}
        \includegraphics[scale=0.25]{Pictures/date_articles/distance1/dating1880.jpg}
        \caption{Dating articles from 1880 with the basic distance}
    \end{minipage}\hfill
    \begin{minipage}[b]{0.3\linewidth}
        \includegraphics[scale=0.25]{Pictures/date_articles/distance1/dating1920.jpg}
        \caption{Dating articles from 1920 with the basic distance}
    \end{minipage}\hfill
    \begin{minipage}[b]{0.3\linewidth}
	\includegraphics[scale=0.25]{Pictures/date_articles/distance1/dating1995_corrected.jpg}
        \caption{Dating articles from 1995 with the basic distance}
    \end{minipage}
    \label{date_d1}
\end{figure}
With the basic metric, the distances look very random. The prediction given by the example on figure \ref{date_d1} is 1843 for articles from 1995.
\subsubsection{Chi Square}
In the same fashion as for the basic distance, here are the distances computed between the subset of articles and each year of the corpus:
\begin{figure}[H]
    \begin{minipage}[b]{0.3\linewidth}
        \includegraphics[scale=0.25]{Pictures/date_articles/chi2/dating1880.jpg}
        \caption{Dating articles from 1880 with the chi-square distance. Prediction is 1890}
    \end{minipage}\hfill
    \begin{minipage}[b]{0.3\linewidth}
        \includegraphics[scale=0.25]{Pictures/date_articles/chi2/dating1920.jpg}
        \caption{Dating articles from 1920 with the chi-square distance. Prediction is 1904.}
    \end{minipage}\hfill
    \begin{minipage}[b]{0.3\linewidth}
	\includegraphics[scale=0.25]{Pictures/date_articles/chi2/dating1995_corrected.jpg}
        \caption{Dating articles from 1995 with the chi-square distance. Prediction is 1858.}
        \label{date_chi2}
    \end{minipage}
\end{figure}
With the \emph{Chi-Square} metric, the distances also look relatively random and there is again no evident prediction in these examples.
\subsubsection{Cosine}

\begin{figure}[H]
    \begin{minipage}[b]{0.3\linewidth}
        \includegraphics[scale=0.25]{Pictures/date_articles/cos/dating1880.jpg}
        \caption{Dating articles from 1880 with the cosine distance. Prediction is 1862.}
    \end{minipage}\hfill
    \begin{minipage}[b]{0.3\linewidth}
        \includegraphics[scale=0.25]{Pictures/date_articles/cos/dating1920.jpg}
        \caption{Dating articles from 1920 with the cosine distance. Prediction is 1889.}
    \end{minipage}\hfill
    \begin{minipage}[b]{0.3\linewidth}
	\includegraphics[scale=0.25]{Pictures/date_articles/cos/dating1995_corrected.jpg}
        \caption{Dating articles from 1995 with the cosine distance. Prediction is 1866}
        \label{date_cos}
    \end{minipage}
\end{figure}
The cosine distance seems to have the same behaviour as the basic and \emph{Chi-Square} distances for dating. Its predictions are also not accurate.
\subsubsection{Cosine with TF-IDF}
Here are the distances computed by the cosine metric with the TF-IDF terms:
\begin{figure}[H]
    \begin{minipage}[b]{0.3\linewidth}
        \includegraphics[scale=0.25]{Pictures/date_articles/cos/dating1880_tfidf.jpg}
        \caption{Dating articles from 1880 with the cosine distance with TF-IDF. Prediction is 1890.}
    \end{minipage}\hfill
    \begin{minipage}[b]{0.3\linewidth}
        \includegraphics[scale=0.25]{Pictures/date_articles/cos/dating1920_tfidf.jpg}
        \caption{Dating articles from 1920 with the cosine distance with TF-IDF. Prediction is 1873.}
    \end{minipage}\hfill
    \begin{minipage}[b]{0.3\linewidth}
	\includegraphics[scale=0.25]{Pictures/date_articles/cos/dating1995_tfidf.jpg}
        \caption{Dating articles from 1995 with the cosine distance with TF-IDF. Prediction is 1865.}
        \label{date_cos_tfidf1}
    \end{minipage}
\end{figure}

The prediction is much more precise than with the previous metrics. As we saw in figure \ref{cos_tfidf}, low distances are very close to the diagonal which means that $d(year1, year2)$ is lower than $1$ only if $year1$ and $year2$ are close to each other. With this property, the prediction of the cosine metric with TF-IDF terms will often be very close to the real year of the articles.
\subsubsection{Kullback-Leibler Divergence}
Figures \ref{ArticleKL-C1880} to \ref{ArticleKL-N1995} show the results for the 3 different years trying to represent the articles with each year :

\begin{figure}[H]
    \begin{minipage}[b]{0.48\linewidth}
        \includegraphics[scale=0.15]{Pictures/date_articles/kullback_leibler/15articles_1880_KL_years_simulate_articles_corrected_without_articles.jpg}
        \caption{KL for 15 articles with OCR correction for year 1880}
        \label{ArticleKL-C1880}
    \end{minipage}\hfill
    \begin{minipage}[b]{0.5\linewidth}
        \includegraphics[scale=0.15]{Pictures/date_articles/kullback_leibler/15articles_1880_KL_years_simulate_articles_without_correction_without_articles.jpg}
        \caption{KL for 15 articles without OCR correction for year 1880}
        \label{ArticleKL-N1880}
    \end{minipage}\hfill
\end{figure}

\begin{figure}[H]
    \begin{minipage}[b]{0.48\linewidth}
        \includegraphics[scale=0.15]{Pictures/date_articles/kullback_leibler/15articles_1920_KL_years_simulate_articles_corrected_without_articles.jpg}
        \caption{KL for 15 articles with OCR correction for year 1920}
        \label{ArticleKL-C1920}
    \end{minipage}\hfill
    \begin{minipage}[b]{0.5\linewidth}
        \includegraphics[scale=0.15]{Pictures/date_articles/kullback_leibler/15articles_1920_KL_years_simulate_articles_without_correction_without_articles.jpg}
        \caption{KL for 15 articles without OCR correction for year 1920}
        \label{ArticleKL-N1920}
    \end{minipage}\hfill
\end{figure}

\begin{figure}[H]
    \begin{minipage}[b]{0.48\linewidth}
        \includegraphics[scale=0.15]{Pictures/date_articles/kullback_leibler/15articles_1995_KL_years_simulate_articles_corrected_without_articles.jpg}
        \caption{KL for 15 articles with OCR correction for year 1995}
        \label{ArticleKL-C1995}
    \end{minipage}\hfill
    \begin{minipage}[b]{0.5\linewidth}
        \includegraphics[scale=0.15]{Pictures/date_articles/kullback_leibler/15articles_1995_KL_years_simulate_articles_without_correction_without_articles.jpg}
        \caption{KL for 15 articles without OCR correction for year 1995}
        \label{ArticleKL-N1995}
    \end{minipage}\hfill
\end{figure}

We took only the direction where the year approximates the articles because it has no sense to do the inverse. Indeed, to approximate a year that contains thousands of words with a subset of articles that contains a few hundred words will lead to a dating that will very often predict the years with less words (in the 19$th$ century or 1998). We should note that for the years 1917 to 1919, there is only one newspaper in the corpus. It explains the peaks in the plots.

We can observe again in these figures that the OCR correction does not help so much for dating. But, we can observe that the \emph{Kullback-Leibler} Divergence is really good to date the articles. Our explanation for these excellent results is that when we remove some articles from a year, the probability that the words are present in this year is higher than in other years. So, even if we remove the articles from the year, it stays hardly attached to the article and when the year approximates the articles, it has a really good matching.

\subsubsection{Out of Place}
In this part, we using \emph{Out of Place} measurement to date the article by two different kind of approaches (i.e., whether use unmatched word or not). See figure \ref{outofplace_1840_all}, \ref{outofplace_1840_match},\ref{outofplace_1960_all}, \ref{outofplace_1960_match}, \ref{outofplace_1995_all} and \ref{outofplace_1995_match} for more details.

\begin{figure}[H]
    \begin{minipage}[b]{0.48\linewidth}
        \includegraphics[scale=0.15]{Pictures/date_articles/outofplace/1840_all.jpg}
        \caption{Out of place for dating article using all words for year 1840}
        \label{outofplace_1840_all}
    \end{minipage}\hfill
    \begin{minipage}[b]{0.5\linewidth}
        \includegraphics[scale=0.15]{Pictures/date_articles/outofplace/1840_partial.jpg}
        \caption{Out of place for dating article using matched words for year 1840}
        \label{outofplace_1840_match}
    \end{minipage}\hfill
\end{figure}

\begin{figure}[H]
    \begin{minipage}[b]{0.48\linewidth}
        \includegraphics[scale=0.15]{Pictures/date_articles/outofplace/1960_all.jpg}
        \caption{Out of place for dating article using all words for year 1960}
        \label{outofplace_1960_all}
    \end{minipage}\hfill
    \begin{minipage}[b]{0.5\linewidth}
        \includegraphics[scale=0.15]{Pictures/date_articles/outofplace/1960_partial.jpg}
        \caption{Out of place for dating article using matched words for year 1960}
        \label{outofplace_1960_match}
    \end{minipage}\hfill
\end{figure}

\begin{figure}[H]
    \begin{minipage}[b]{0.48\linewidth}
        \includegraphics[scale=0.15]{Pictures/date_articles/outofplace/1995_all.jpg}
        \caption{Out of place for dating article using all words for year 1995}
        \label{outofplace_1995_all}
    \end{minipage}\hfill
    \begin{minipage}[b]{0.5\linewidth}
        \includegraphics[scale=0.15]{Pictures/date_articles/outofplace/1995_partial.jpg}
        \caption{Out of place for dating article using matched words for year 1995}
        \label{outofplace_1995_match}
    \end{minipage}\hfill
\end{figure}

If we only look at the result of year $1840$, we can say that it is a good metric for its high-accuracy of prediction. However, when tracking the condition on other years, the result is really odd since all of them give a same plot if we just classify them by eyes. Check its detailed information, there exists a slight difference between different pairs of article and years for the \emph{Out of Place} measurement. However, we cannot deny that it works bad on dating article.

Returning to the formula of \emph{Out of Place} measurement and also considering the size of article, this odd phenomenon then can be explained. For dating article, we only randomly select 15 articles whose size is too small to affect the rank of words that appears in the year data even though we have already removed word's frequency of article from the year it belongs to. As a result, noises are introduced to the final result or useful information are missing.

If we use all words, due to the big gap between the size of article and years, valuable information will be hidden by large volume noises which explains the unchangeable phenomenon on the distance of different article and years. If we only use matched words, for articles that from different years, the matched words are greatly limited by the year data, thus the computation process will always happen on the similar article dataset and year dataset if article data varies but year data remains the same. That is why the \emph{Out of Place} measurement met some changes among part of relationships but still remain a similar trend.

Thus, as explained above, the \emph{Out of Place}, at least for the function of dating article, is not a good metric.


\subsubsection{Punctuation and sentences length}
As the statistics on punctuation and sentences tend to change between years (especially for the sentences lengths and number of commas), we tried to implement a metric with those statistics. \\

Those statistics are extracted for each year and combined in a single entry by year. For the metric, we take a sample of articles from one year, we merge them, as it was one big article, we compute the statistics on this article and we use a simple euclidean distance to compute the distance with each year. The result for a sample of 15 random articles from 1925 is observable in figure \ref{punct_metric_1925}.

\begin{figure}[H]
	\centering
    \includegraphics[scale=0.5]{Pictures/date_articles/punctuation/graph.png}
    \caption{Distance between a sample of 15 random articles from 1925 and each year}
    \label{punct_metric_1925}\hfill
\end{figure}

As we can see, it does not work that well, but with more time, we could have used machine learning methods instead of a simple euclidean distance, we think it could be useful to combine the result obtained them with other distances to enhance the classification.

\subsection{Cross Validation}
To have stronger values when we date articles, we wrote a script that runs several times each metrics for a different subsets of article of the same year. The idea here is to have an average on the error of each metric to be able to compare them together. We obtained the following results for random years.\\

\textbf{Mean error for different metrics for 15 articles in year 1843 with 15 iterations}\\
\begin{tabular}{p{3cm} p{5cm}}
Distance1 :& 0\\
Cosine :& 10.73333333333333333333\\
Cosine-TFIDF :& .40000000000000000000\\
Chi-Square :& 112.40000000000000000000\\
Kullback-Leibler :& 11.33333333333333333333\\
OutOfPlace :& 3.00000000000000000000\\
Punctuation :& 33.13333333333333333333\\
\end{tabular}\\
 
\textbf{Mean error for different metrics for 15 articles in year 1854 with 15 iterations}\\
\begin{tabular}{p{3cm} p{5cm}}
    Distance1 :& 8.00000000000000000000\\
    Cosine :& 9.86666666666666666666\\
    Cosine-TFIDF :& 0\\
    Chi-Square :& 111.86666666666666666666\\
    Kullback-Leibler :& 0\\
    OutOfPlace :& 8.00000000000000000000\\
    Punctuation :& 15.80000000000000000000\\
\end{tabular}\\
 
\textbf{Mean error for different metrics for 15 articles in year 1888 with 15 iterations}\\
\begin{tabular}{p{3cm} p{5cm}}
    Distance1 :& 42.00000000000000000000\\
    Cosine :& 15.13333333333333333333\\
    Cosine-TFIDF :& 0\\
    Chi-Square :& 62.40000000000000000000\\
    Kullback-Leibler :& .26666666666666666666\\
    OutOfPlace :& 42.00000000000000000000\\
    Punctuation :& 32.80000000000000000000\\
\end{tabular}\\
 
\textbf{Mean error for different metrics for 15 articles in year 1905 with 15 iterations}\\
\begin{tabular}{p{3cm} p{5cm}}
    Distance1 :& 65.80000000000000000000\\
    Cosine :& 17.13333333333333333333\\
    Cosine-TFIDF :& .06666666666666666666\\
    Chi-Square :& 48.40000000000000000000\\
    Kullback-Leibler :& 3.06666666666666666666\\
    OutOfPlace :& 59.00000000000000000000\\
    Punctuation :& 21.06666666666666666666\\
\end{tabular}\\
 
\textbf{Mean error for different metrics for 15 articles in year 1918 with 15 iterations}\\
\begin{tabular}{p{3cm} p{5cm}}
    Distance1 :& 78.93333333333333333333\\
    Cosine :& 4.73333333333333333333\\
    Cosine-TFIDF :& 0\\
    Chi-Square :& 36.33333333333333333333\\
    Kullback-Leibler :& 2.66666666666666666666\\
    OutOfPlace :& 72.00000000000000000000\\
    Punctuation :& 30.86666666666666666666\\
\end{tabular}\\
 
\textbf{Mean error for different metrics for 15 articles in year 1934 with 15 iterations}\\
\begin{tabular}{p{3cm} p{5cm}}
    Distance1 :& 65.60000000000000000000\\
    Cosine :& 17.46666666666666666666\\
    Cosine-TFIDF :& .80000000000000000000\\
    Chi-Square :& 128.93333333333333333333\\
    Kullback-Leibler :& .33333333333333333333\\
    OutOfPlace :& 88.00000000000000000000\\
    Punctuation :& 37.66666666666666666666\\
\end{tabular}\\
 
\textbf{Mean error for different metrics for 15 articles in year 1954 with 15 iterations}\\
\begin{tabular}{p{3cm} p{5cm}}
    Distance1 :& 44.00000000000000000000\\
    Cosine :& 24.60000000000000000000\\
    Cosine-TFIDF :& 0\\
    Chi-Square :& 0\\
    Kullback-Leibler :& .66666666666666666666\\
    OutOfPlace :& 108.00000000000000000000\\
    Punctuation :& 48.06666666666666666666\\
\end{tabular}\\
 
\textbf{Mean error for different metrics for 15 articles in year 1984 with 15 iterations}\\
\begin{tabular}{p{3cm} p{5cm}}
    Distance1 :& 14.00000000000000000000\\
    Cosine :& 7.20000000000000000000\\
    Cosine-TFIDF :& 1.66666666666666666666\\
    Chi-Square :& 3.73333333333333333333\\
    Kullback-Leibler :& 0\\
    OutOfPlace :& 138.00000000000000000000\\
    Punctuation :& 81.80000000000000000000\\
\end{tabular}\\

We observe that two metrics are in general better than the others. These two metrics are the \emph{Kullback-Leibler} Divergence and the Cosine with TF-IDF. Indeed, as explained in section \ref{metrics}, these two metrics are good and we know why. The other metrics are not so bad, even the basic distance which has surprisingly an error of 0 when dating articles from 1843. We think this distance was just lucky with the subset of articles because for the other years, the basic distance is clearly not the best one. For the other metrics, we can observe that they vary depending on the year or certainly on the subset of articles.\\

To judge which metric is the best one, we can simply compute the mean error over all our samples. As it takes a lot of time to compute 15 times each metrics, we can't compute it for each year. Here is the overall mean over all the years we tested :\\

\textbf{Mean error for different metrics for 15 articles in the years selected above with 15 iterations per year}\\
\begin{tabular}{p{3cm} p{5cm}}
    Distance1 :& 39.7875\\
    Cosine :& 13.35875\\
    Cosine-TFIDF :& 0.36675\\
    Chi-Square :& 63.008625\\
    Kullback-Leibler :& 2.292\\
    OutOfPlace :& 64.75\\
    Punctuation :& 37.65025\\
\end{tabular}\\

We can observe that, as expected, Cosine with TF-IDF distance and \emph{Kullback-Leibler} Divergence are the best ones.\\

Finally, we made some tests to compare, for a given year, if the number of selected articles to date changes a lot the results of the metrics. We chose to run for a given year (1934) the dating of articles with subsets of articles of sizes 5, 15, 30, 45, 60, 75 and 90. The results are shown in the following table (the smallest distance is colored in green, the biggest in red in table \ref{table_comparison_size_articles}).
\begin{center}
    \begin{table}[h!]
        \begin{tabular}{ | p{3.3cm} | p{1.4cm} | p{1.4cm} | p{1.4cm} | p{1.4cm} | p{1.4cm} | p{1.4cm} | p{1.4cm} | }
            \hline
            \textbf{\# of articles} & \textbf{5} & \textbf{15} & \textbf{30} & \textbf{45} & \textbf{60} & \textbf{75} & \textbf{90}\\
            \hline
            \hline
            \textbf{Basic Distance} & \textcolor{green}{64} & \textcolor{green}{64} & \textcolor{red}{190.0667} & 65.4 & 66.8 & & \\
            \hline
            \textbf{Cosine} & 35 & \textcolor{green}{18.7334} & \textcolor{red}{270.2667} & 135.8667 & 265.6 & & \\
            \hline
            \textbf{Cosine-TFIDF} & 2.5334 & \textcolor{green}{0.6667} & 257.8667 & 128.9334 & \textcolor{red}{258.2667} & &  \\
            \hline
            \textbf{Chi-Square} & 2.1334 & \textcolor{green}{0} & \textcolor{red}{257.8667} & 128.9334 & \textcolor{red}{257.8667} & & \\
            \hline
            \textbf{Kullback-Leibler} & \textcolor{green}{0.0667} & \textcolor{green}{0.6667} & \textcolor{red}{134.6667} & 5.7334 & 11.4667 & & \\
            \hline
            \textbf{Out of Place} & 89.2 & \textcolor{green}{88.2667} & 334.6667 & 211.2 & \textcolor{red}{334.4} & & \\
            \hline
            \textbf{Punctuation} & 46.6 & \textcolor{red}{52.4667} & 46 & 28.667 & \textcolor{green}{33.3337} & & \\
            \hline
        \end{tabular}
        \caption{Mean error distance for the year 1934 with different numbers of articles for 15 iterations}
        \label{table_comparison_size_articles}
    \end{table}
\end{center}

We observe that in general, more we have articles, worst is the dating. First, we have to note that more we take articles, more we remove words in the article's year which can make the number of words in this year consequently smaller than other years. It could be an explanation why taking a lot of articles to date them is not necessarily a good thing. Nevertheless, we observe that the punctuation distance prefers a bigger set of articles, because it can have a better approximation of the sentences length, number of commas and so on for the article's year. For the other metrics, the best dating is reached between with 15 articles in the subset. With only 5 articles, the subset should not contain enough words to be able to date as precisely as with 15 articles. Finally, we can observe that the worst case is with 30 articles, which is quite strange. We explain that with the fact that the articles.... RECOMPUTE DATA BECAUSE OF ERRORS !

