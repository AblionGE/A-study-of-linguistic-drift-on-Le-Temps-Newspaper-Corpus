The dating is done by computing the distance between a small set of articles and each year. The prediction is the year for which the distance is the smallest.

\begin{figure}[H]
    \begin{minipage}[b]{0.3\linewidth}
        \includegraphics[scale=0.25]{Pictures/date_articles/distance1/dating1880.jpg}
        \caption{Dating articles from 1880 with the basic distance}
    \end{minipage}\hfill
    \begin{minipage}[b]{0.3\linewidth}
        \includegraphics[scale=0.25]{Pictures/date_articles/distance1/dating1920.jpg}
        \caption{Dating articles from 1920 with the basic distance}
    \end{minipage}\hfill
    \begin{minipage}[b]{0.3\linewidth}
	\includegraphics[scale=0.25]{Pictures/date_articles/distance1/dating1995_corrected.jpg}
        \caption{Dating articles from 1995 with the basic distance}
    \end{minipage}
    \label{date_d1}
\end{figure}
With the basic metric, the distances look very random. The prediction given by the example on figure \ref{date_d1} is 1843 for articles from 1995.