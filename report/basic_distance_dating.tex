The dating is done by computing the distance between a small set of articles and each year. The prediction is the year for which the distance is the smallest.
\begin{figure}[H]
	\centering
        \includegraphics[scale=0.15]{Pictures/date_articles/distance1/dating1995_corrected.jpg}
        \caption{Dating articles from 1995 with the basic distance}
        \label{date_d1}
\end{figure}
With the basic metric, the distances look very random. The prediction given by the example on figure \ref{date_d1} is 1843 for articles from 1995.