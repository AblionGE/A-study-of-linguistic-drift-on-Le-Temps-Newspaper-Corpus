To conclude this report, we can say that there exists a linguistic drift in our newspaper's corpus. Indeed, a straightforward statement is that we use more vocabulary to express the same ideas and we do shorter sentences. We also observed that the gain on words through is more important than words disappearances.

About the OCR correction, we saw that for some metrics the correction doesn't have any influence but for other ones (in particular for cosine distance), it helps to improve the observation of the linguistic drift. Then, with the TF-IDF values, it is again the cosine metric that benefits from the importance of uncommon words. For the other metrics, it is less useful.

Then, when we tried to date articles, we obtained some really good results using the \emph{Kullback-Leibler} Divergence and the cosine distance with TF-IDF. Indeed, with those two metrics, we were able to date a subset of articles with less than 10 years error in general what could be considered as an indicator of the goodness of these metrics. With the other metrics, the error is a bit bigger, but they can have sometimes some good results due to some specific reasons (like the number of words in the year).

The synonym approach has revealed interesting points such that words that have similar meaning evolve in a different manner through time and also by their popularity. But we need to use this with precaution because of the several meaning of words (e.g. "caisse"). 

Finally, with the topics, we found some interesting ones for which we were able to found some properties using the \emph{Kullback-Leibler} Divergence and the cosine distance. Indeed, with the KL divergence, we saw that years in general are not divergent to each other and with cosine we saw that years that are close have a small distance between them.