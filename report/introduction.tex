\section{Introduction}

\subsection{Problem Statement}
Intuition helps us think that, through time, language is changing because of the variation of the vocabulary but also in the way words of this vocabulary are used.
Along with discoveries and civilization evolution, some words disappear and others emerge. Moreover, the 21\textsuperscript{st} century is characterized by secondary source documents that influence us in our way of expressing ourselves. Thus, our study tries to define the features that could prove that there is a linguistic drift and to quantify the changes with a metric.
Our dataset is a french articles corpus from 1840 to 1998 from two Swiss newspapers. We are provided with xml files of OCRized articles. The fact that there is less articles during the first years is sometimes to be considered, as is the fact that for years 1917 to 1919 we just have one newspaper's articles and that in 1998, we just have the articles of a few months.\\

Within this project, we want to observe the evolution of the language : \enquote{Do some words appear and then disappear ?} \enquote{How much different words do we need to cover a certain percentage of the language ?} \enquote{Is the correction of the OCR useful to observe the linguistic drift ?} \enquote{Does giving more weight to uncommon words improve the drift ?} \enquote{Is it possible to date an article with precision using only its vocabulary ?} \enquote{Are topics helpful for observing the linguistic drift ?} These are the questions we want to answer through our project.\\

This report is structured as follows : first, in section \ref{statistics} we present some statistics about the corpus using the number of words, the sentences, their length, the use of punctuation. Secondly, in section \ref{metrics}, we present and explain different metrics we used to try to observe a linguistic drift. Then, in section \ref{dating}, we try to date articles using the metrics presented in section \ref{metrics}. After that, in section \ref{topics}, we present some topics we found and we try to date articles according to topics. Finally, in section \ref{conclusion}, we answer the questions stated in this introduction.
