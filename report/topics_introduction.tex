We wanted to see how is the evolution of the language inside a specific topic of the corpus (e.g. swiss politics, sport, art,...).	
Our idea was that a topic clustering would help us capture the linguistic drift more precisely when comparing two articles that talk about the same subject.
This new approach would give us different results.\\
For topic clustering, we chose to use \emph{Latent Dirichlet Allocation} (LDA) implemented in spark since spark 1.3.0. 
The algorithms take as input a matrix of article and 1-gram occurrences and cluster the articles into a specified number \textit{n} of topics. We could also set the parameters for topic distribution over terms $\beta$ and articles distribution over topics $\alpha$. Empirical results led us to set the $\beta$ to 17, a relatively high value because articles contain lot of stop words that could have too much weight for lower values. In the same way, we set $\alpha$ to 1.1, the minimum value, as articles in a newspaper would rather belong to a single topic. For a number of topics equal to \textit{n} equal to 15.\\
The LDA algorithm gave us as output the following sample of topic(the words are giving in the decreasing order of weight to describe a topic):\\
TOPIC: Sport\\
match, equipe, club, championnat\\
TOPIC: Art\\
oeuvre, histoire, auteur, oeuvres\\
TOPIC: Undefined\\
bien, sans, homme, fait\\
TOPIC: Politics\\
politique, gouvernement, conseil, parti
TOPIC: Banque, Stock\\
banque, action, bourse, paris\\
TOPIC: Governement\\
ministre, etats, unis, gouvernement\\
TOPIC: Swiss Politic\\
conseil, president, etat, canton\\
TOPIC: Swiss Market\\
dutch, nestle, shell, steel, pacific\\

